% define path information
\newcommand{\assets}{../../assets/}
\newcommand{\tables}{\assets tables/}
\newcommand{\figures}{\assets figures/}
\newcommand{\static}{\assets static/}

\def\anonymized{0} % uncomment to compile the non-anonymized version of the document
% \def\anonymized{1} % uncomment to compile the anonymized version of the document

\documentclass[11pt]{article}
\usepackage[utf8]{inputenc}
\usepackage[english]{babel}
\usepackage[a4paper, margin=1in]{geometry}
\usepackage[normalem]{ulem}
\usepackage{
  amsmath,amsthm,amscd,amsfonts,amssymb,
  threeparttable,booktabs,pdflscape,tabularray,
  graphicx,float,hyperref,
  lineno,fancyhdr,
  natbib,
  setspace,siunitx,
  appendix, 
  codehigh
}
\UseTblrLibrary{booktabs}
\UseTblrLibrary{siunitx}
\newcommand{\tinytableTabularrayUnderline}[1]{\underline{#1}}
\newcommand{\tinytableTabularrayStrikeout}[1]{\sout{#1}}
\NewTableCommand{\tinytableDefineColor}[3]{\definecolor{#1}{#2}{#3}}

\newtheorem{theorem}{Theorem}
\newtheorem{proposition}{Proposition}
\newtheorem{lemma}{Lemma}
\newtheorem{corollary}{Corollary}
\newtheorem{conjecture}{Conjecture}
\theoremstyle{definition}
\newtheorem{definition}{Definition}
\newtheorem{assumption}{Assumption}
\newtheorem{axiom}{Axiom}
\newtheorem{hypothesis}{Hypothesis}
\theoremstyle{remark}
\newtheorem{remark}{Remark}


\title{
  On the Reproducibility of Plausible Results
  % \thanks{Acknowledgements}
}
\author{
  Alex Lambda
  \thanks{
    Institute for Advanced Templates, {\tt \href{mailto:alex.lambda@example.com}{alex.lambda@example.com}.}
  }
}
\date{\today}

% ---- Publication ready manuscript has a fancy page header ----
\if\anonymized0
\pagestyle{fancy}
\fancyhf{}
\lhead{On the Reproducibility of Plausible Results}
\rhead{
  Lambda
}
\cfoot{\thepage}
\setlength{\headheight}{14pt}
\fi

% ---- Anonymized manuscript specific layout ---
\if \anonymized1
\onehalfspacing % wider line spacing
\linenumbers % add line numbers
\fi

\begin{document}
\sloppy % avoid overfull hboxes

\if\anonymized0
\maketitle
%TC:ignore
\begin{abstract}
This paper examines the reproducibility of plausible empirical results using a stylized dataset on student performance. We estimate a linear model relating exam outcomes to study behavior, attendance, sleep, and prior achievement, and use the exercise to demonstrate a fully reproducible research workflow. While the data and findings are intentionally simplified, the analysis illustrates how standard empirical results can be documented, replicated, and extended with minimal friction. The paper serves as a minimal working example rather than a substantive contribution.

\end{abstract}
%TC:endignore
\fi

\section{Introduction}
\label{sec:introduction}

Reproducibility is now widely recognized as a central concern in empirical research. Recent work by \citet{lambda2024defaults} emphasizes that many irreproducible results stem not from flawed methods, but from undocumented defaults, missing metadata, and fragile workflows. Here, we argue that even \emph{entirely plausible} results may \emph{fail to reproduce} when computational environments are not explicitly specified.

This paper follows that perspective. Using a simple dataset on student outcomes \citep{student_exam_scores_kaggle}, we estimate a conventional regression model and present the results in tabular and graphical form (Section \ref{sec:results}). The goal is not to uncover new empirical insights, but to provide a compact, transparent example that can be reproduced end-to-end with minimal effort.

\section{Results}
\label{sec:results}

We estimate the following linear model:
\begin{equation}
  \text{score}_i = \beta_0 + \beta_1 \text{hours\_studied}_i + \beta_2 \text{sleep\_hours}_i
  + \beta_3 \text{attendance}_i + \beta_4 \text{previous\_scores}_i + \varepsilon_i,
\end{equation}
where $\varepsilon_i$ denotes an idiosyncratic error term. Table~\ref{tab:regression} reports the estimated coefficients, and Figure~\ref{fig:coefplot} visualizes them with 95\% confidence intervals.

\begin{table}[ht]
  \centering
  \label{tab:regression}
  \caption{Regression of Exam Scores on Study Inputs}
  \input{\tables table.tex}
\end{table}

\begin{figure}[ht]
  \centering
  \includegraphics{\figures figure.pdf}
  \caption{Coefficient Estimates with Confidence Intervals}
  \label{fig:coefplot}
\end{figure}

The results are broadly consistent with intuition. In particular:
\begin{enumerate}
  \item prior performance is strongly predictive of current exam outcomes,
  \item attendance is positively associated with scores, and
  \item study time has a modest but precisely estimated effect.
\end{enumerate}

For completeness, we note three features of the empirical setup:
\begin{itemize}
  \item {\bf Measurement.} All variables are measured at the individual level.
  \item {\bf Illustration.} The sample is purely illustrative.
  \item {\bf Linearity.} The specification is intentionally linear.
\end{itemize}

\begin{figure}[ht]
  \centering
  \includegraphics[width=0.4\textwidth]{\static dk.jpg}
  \caption{An External Image Included for Demonstration Purposes}
  \label{fig:donkeykong}
\end{figure}

Figure~\ref{fig:donkeykong} provides an external visual reference included solely to demonstrate the handling of figures.

% ---- Bibliography ----
\newpage
%TC:ignore
\bibliographystyle{aer}
\bibliography{\assets references}
%TC:endignore

% ---- Appendix ----

% \newpage
%TC:ignore
% \appendix
% \appendixpage
% \addappheadtotoc
% Appendix content
%TC:endignore

\end{document}